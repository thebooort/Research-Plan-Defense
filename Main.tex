%%%%%%%%%%%%%%%%%%%%%%%%%%%%%%%%%%%%%%%%%
% Template modified from Ratheesh Ravindran
%%%%%%%%%%%%%%%%%%%%%%%%%%%%%%%%%%%%%%%%%
\documentclass[12pt]{article}
\usepackage[spanish]{babel}
\usepackage[utf8x]{inputenc}
\usepackage{amsmath}
\usepackage{graphicx}
\graphicspath{{Images/}}
\usepackage[colorinlistoftodos]{todonotes}
\usepackage{hyperref}

%\usepackage{fancyhdr}
%\pagestyle{fancy}
\begin{document}
\begin{titlepage}
  \newcommand{\HRule}{\rule{\linewidth}{0.1mm}}
  \center % Center everything on the page
 
  % ---------------------------------------------------------------------------------
  % HEADING SECTIONS (
  % ---------------------------------------------------------------------------------
  \textsc{\Large Plan de Investigación: Tesis Doctoral}\\[0.5cm]
  % ---------------------------------------------------------------------------------
  % TITLE SECTION
  % ---------------------------------------------------------------------------------

  \HRule \\[0.4cm]
  { \huge \bfseries Análisis estandarizado de redes complejas basado
    en metaheurísticas a varios niveles
  }\\[0.1cm] % Title of your Homework/assignment
  \HRule \\[1.5cm]
 
  % ---------------------------------------------------------------------------------
  % AUTHOR SECTION
  % ---------------------------------------------------------------------------------

  \begin{minipage}{0.6\textwidth}
    \begin{flushleft} \large

      \emph{\bf{Autor:}}\\
      Bartolomé Ortiz Viso

      \vspace*{20px}

      \emph{\bf{Director:}} \\
      Juan Julián Merelo Guervós
    \end{flushleft}
  \end{minipage}\\[1cm]
  \large{\bf{Programa de Doctorado en Tecnologías de la Información y
      la Comunicación}} \vspace*{50px}

  {\large
    \today}\\[1cm] % Date, change the \today to a set date if you want to be precise
  \vspace*{10px}
  \includegraphics[width=12cm,height=4cm]{logougr.jpg}% \\[0.5cm] %
  \vfill % Fill the rest of the page with white-space

\end{titlepage}

% ---------------------------End of first
% page----------------------------------

\tableofcontents          % Required
\listoffigures \listoftables
\newpage

% ---------------------------------------------------------------------------------
% Document start
% ---------------------------------------------------------------------------------


\section{Abstract}
\bf{Actualmente los grafos son utilizados en el estudio de muchos
fenomenos en biología, tales como las rutas metabólicas o la
regulación génica.

Sin embargo, no hay herramientas que permitan modelizar, analizar y
visualizar estos sistemas de forma unificada. Carecemos de una
metodologia para encontrar, a partir de la evidencia, el modelo
reticular subyacente que permita entender mejor el sistema.

Trataremos de aplicar un enfoque análogo al aprendizaje profundo
(automatización y procesamiento de la información en diferentes capas,
generando progresivamente un modelo más preciso), a problemas
resolubles o modelizables mediante grafos, de forma que se automaticen
desde el diseño del grafo hasta la métrica escogida para evaluar la
bondad de la solución del problema.

Nuestro objetivo es crear una metodología unificada de
preprocesamiento, procesamiento, y análisis de redes biológicas. Así
como, utilizar técnicas de "transferlearning" para trasaladar
resultados entre redes.}

\section{Antecedentes}


\section{Metodología}


\section{Objetivos}


\section{Medios y Planificación Temporal}


\bibliographystyle{apa} \bibliography{research_plan.bib}

\appendix
\section{Apéndice}


\end{document}
